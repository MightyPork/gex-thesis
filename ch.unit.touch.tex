\section{Touch Sense Unit}

The touch sensing unit provides an access to the \gls{TSC} peripheral, explained in section \ref{sec:theory-touch}. The unit configures the \gls{TSC} and reads the output values of each enabled touch pad. Additionally, a threshold-based digital input function is implemented to make the emulation of touch buttons easier. The hysteresis and debounce time can be configured in the configuration file or set using a command. The threshold of individual pads must be set using a command.

\subsection{Touch Sense Configuration}

\begin{inicode}
[TOUCH:touch@11]
# Pulse generator clock prescaller (1,2,4,...,128)
pg-clock-prediv=32
# Sense pad charging time (1-16)
charge-time=2
# Charge transfer time (1-16)
drain-time=2
# Measurement timeout (1-7)
sense-timeout=7

# Spread spectrum max deviation (0-128,0=off)
ss-deviation=0
# Spreading clock prescaller (1,2)
ss-clock-prediv=1

# Optimize for interlaced pads (individual sampling with others floating)
interlaced-pads=N

# Button mode debounce (ms) and release hysteresis (lsb)
btn-debounce=20
btn-hysteresis=10

# Each used group must have 1 sampling capacitor and 1-3 channels.
# Channels are numbered 1,2,3,4

# Group1 - 1:A0, 2:A1, 3:A2, 4:A3
g1_cap=
g1_ch=
# Group2 - 1:A4, 2:A5, 3:A6, 4:A7
g2_cap=
g2_ch=
# ...
\end{inicode}


\subsection{Touch Sense Events}

\begin{cmdlist}
    0 & \cname{BUTTON\_CHANGE}
    The binary state of some of the capacitive pads with button mode enabled changed.
    & \begin{cmdpld}
        \cfield{u32} binary state of all channels
        \cfield{u32} changed / trigger-generating channels
    \end{cmdpld} \\
\end{cmdlist}

%\newpage
\subsection{Touch Sense Commands}

\begin{cmdlist}
    0 & \cname{READ}
    Read the raw touch pad values (lower indicates higher capacitance). Values are ordered by group and channel.
    & \begin{cmdreq}
        \cfield{u16[]} raw values
    \end{cmdreq} \\

    1 & \cname{SET\_BIN\_THR}
    Set the button mode thresholds for all channels. Value 0 disables the button mode for a channel.
    & \begin{cmdreq}
        \cfield{u16[]} thresholds
    \end{cmdreq} \\

    2 & \cname{DISABLE\_ALL\_REPORTS}
    Set thresholds to 0, disabling the button mode for all pads.
    & \\

    3 & \cname{SET\_DEBOUNCE\_TIME}
    Set the button mode debounce time (used for all pads with button mode enabled).
    & \begin{cmdreq}
        \cfield{u16} debounce time (ms)
    \end{cmdreq} \\

    4 & \cname{SET\_HYSTERESIS}
    Set the button mode hysteresis.
    & \begin{cmdreq}
        \cfield{u16} hystheresis
    \end{cmdreq} \\

\end{cmdlist}





