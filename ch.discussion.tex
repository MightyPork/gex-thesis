\chapter{Discussion}

The objective of this work was to develop an extensible platform allowing the user to interact with hardware modules and circuitry from high-level applications running on their \gls{PC}, in other words, to ``bring the Raspberry Pi \gls{GPIO} header to the \gls{PC}''. It was important to design the platform in such a way that a user without much experience with embedded electronics and low level circuitry could still use it with a reasonable level of comfort and confidence.

\section{Solution Summary}

We based the \textit{GEX} platform around STM32 microcontrollers from ST Microelectronics and developed the initial version for the STM32F072. This \gls{MCU} was chosen for its native \gls{USB} Full Speed support, its interesting set of peripheral blocks, and because it is a relatively recent model in the STM32F series, expecting it to be mature and without major flaws. A wireless connection was planned together with \gls{USB} and hardware \gls{UART}, for which we chose the nRF24L01+ transceiver from Nordic Semiconductor.

In total, three custom hardware modules have been developed and realized for the project: GEX Hub, GEX Zero, and the wireless gateway. GEX Hub verified the designed schematic and general functionality of the module. GEX Zero is more advanced and compact than GEX Hub, also offering wireless connectivity for field experiments; we re-used the form factor of the Raspberry Pi Zero for this board, which makes it compatible with many existing enclosures and add-on boards. In addition to custom hardware, the GEX firmware may be flashed to STM32 development boards (``STM32F072B-DISCO'').

\noindent
The initial version of the project supports the following hardware buses and functions:

\begin{itemize}[itemsep=1pt]
	\item \textbf{Digital input} with asynchronous pin change detection
	\item \textbf{Digital output} with pulse generation support
	\item \textbf{Digital communication interfaces:}	
\begin{itemize}
\item \textbf{\gls{SPI}} with up to 16 slaves and multicast support
\item \textbf{\gls{I2C}} master
\item \textbf{\gls{USART}} with optional RS-485 driver support
\item \textbf{1-Wire} master, implementing the \gls{ROM} Search algorithm
\end{itemize}
	\item \textbf{Frequency, pulse width, and duty cycle measurement}
	\item \textbf{Pulse counting}
	\item \textbf{\gls{PWM} signal generation}	
	\item \textbf{Multi-channel shift register loading}
	\item \textbf{NeoPixel RGB \gls{LED} strip control}
	\item \textbf{Capacitive touch sensing}
	\item \textbf{Analog signal acquisition}
		\begin{itemize}
			\item Immediate or averaged voltage measurement
			\item Isochronous sampling: fixed sample count, or a continuous stream
			\item Threshold-based triggering with a pre-trigger buffer
		\end{itemize}
	\item \textbf{Analog signal generation}
		\begin{itemize}
			\item Sine, triangle, sawtooth, or rectangle waveform generation using \gls{DDS}
			\item \gls{DC} level output
		\end{itemize}
\end{itemize}


We developed a transaction-based binary protocol that is used by all three communication interfaces. Software libraries implementing this protocol have been developed in programming languages C and Python. It has been tested and verified that the Python library may be integrated into MATLAB scripts, should the user need MATLAB, e.g., for data processing or visualization.

To make the devices easy to reconfigure, the configuration options---such as, which features to enable and what pins to use for them---are presented by entries in INI configuration files. The user can edit these files in a virtual \gls{USB} mass storage device with an emulated FAT16 file system. Because the emulation relies on a certain pattern of behavior from the host, it may not work reliably in all cases. This is mitigated by making the files accessible also through the communication interface.

\section{Results and Possible Applications}

The device we designed, and the accompanying software stack, make a practical tool for research and development in the field of embedded electronics, and may also be used as a learning aid to demonstrate the timings and format of the various supported hardware buses. The inclusion of analog signal acquisition and generation features makes it possible to measure the analog characteristics of electronic components, or acquire data from physical phenomena. 

The use in automation offers itself thanks to the programmatic control from high level programming languages on the \gls{PC}; this has an advantage over programming the embedded firmware itself in that the user can focus on the behavior of the device they are developing, rather than having to deal with low level implementation details, such as memory management, or interrupt priority configuration. An inherent downside of this universal platform lies in greater latencies and lower performance than could be achieved with a custom embedded system.

As the GEX platform is open source, with all source code and hardware materials made available on GitHub~\cite{gex-gh}, our solution may be freely expanded and adapted to support other hardware platforms, to add new features, or to further optimize its performance.

\section{Comparison with Existing Solutions}

The integration of the \gls{PC} with low-level hardware is an appealing idea that has been explored in several alternative solutions, which we listed in \cref{sec:prior_art}. It is our belief that the GEX project offers enough unique features and qualities to find its place among the competition.

Compared to the Raspberry Pi, GEX is less powerful and cannot be used as a stand-alone device without the host \gls{PC}. However, it is more straightforward to use, the required configuration is minimal, and it offers features not available in the Raspberry Pi, such as the \gls{ADC} and \gls{DAC}.

The Bus Pirate excels as a bus analyzer and general purpose device and implements features we did not explore in this project, while having much fewer \gls{GPIO} pins, a lower resolution of the \gls{ADC}, and completely misses a \gls{DAC}. The two projects serve a slightly different purposes and appear to be complementary.

It is clear that GEX does not pose a real threat to professional tools in terms of performance, but even at the prototype cost it is significantly more affordable. By integrating many diverse features, GEX can replace several instruments at once, allowing the creation of more advanced measurement systems in a compact form.

\section{Limitations}

While the objective of the work was achieved, we are aware of several limitations that must be acknowledged so that they can be addressed in future work.

The chosen STM32F072 microcontroller proved sufficient for the verification of the design, and in many cases provides sufficient performance. Future expansion of the project is, unfortunately, limited by its flash and \gls{RAM} capacity, which are already used at about 85\,\%, based on the size of the binary image and the amount of allocated memory. Further, the STM32F072 has a Cortex-M0 core without any hardware support for floating point arithmetics, making any calculations with those numbers slow and requiring additional library functions, further increasing the firmware image size. The modules' power consumption has not been measured or optimized, but the inclusion of a wireless interface predisposes them to be used in battery-powered applications; the microcontroller choice and the power supply design may need to be revised for efficient battery operation.

Another limitation concerns the support software; our client libraries have not been tested on MS Windows, with Linux being the main development platform. The C library uses POSIX \gls{API} and UNIX-specific \gls{API} to manage the serial port, which is not portable. The Python library should work on MS Windows, provided libUSB is installed correctly. Further, on MS Windows prior to version 10, the virtual COM port connection requires the ``STM32 Virtual COM port driver'' to be manually installed and assigned in the Device Manager. 

Lastly, our solution was designed with the expectation that the users are familiar with programming, and the requirement to develop a control script or application would not pose a problem. However, that is not universally true, and some users who might benefit from the platform will not be able to use it without a more user-friendly interface.

\section{Future Development}

Future development of the project should focus on expanding the number of supported hardware platforms in order to overcome the limitations of the used microcontroller model. In particular, the STM32L series of low-power devices should be investigated, as it may be a better choice when a battery-based operation is needed. The hardware implementation shall further be revised to maximize its power efficiency.

The client libraries need to be tested on other operating systems, and the C library may be expanded to provide a higher level of comfort to the user. The client library could further be reimplemented in more programming languages, such as Java or \CS, and graphical applications could be developed to make GEX more approachable to users less familiar with programming.

GEX Zero, with the Raspberry Pi Zero form factor, introduced the possibility of using Raspberry Pi add-on boards. It would be interesting to test the compatibility with more such existing boards, and further try designing custom ones. An add-on board could, for instance, add access to the \gls{DALI} bus, RS-485, \gls{CAN}, or Ethernet.





