\section{I2C Unit}

The I2C unit provides access to one of the microcontroller's I2C peripherals. It can be configured to use either of the three speeds (Standard, Fast and Fast+) and supports both 10-bit and 7-bit addressing. 10-bit addresses can be used in commands by setting their highest bit (0x8000), as a flag to the unit. 

\subsection{I2C Configuration}

\begin{inicode}
[I2C:d@4]
# Peripheral number (I2Cx)
device=1
# Pin mappings (SCL,SDA)
#  I2C1: (0) B6,B7    (1) B8,B9
#  I2C2: (0) B10,B11  (1) B13,B14
remap=0

# Speed: 1-Standard, 2-Fast, 3-Fast+
speed=1
# Analog noise filter enable (Y,N)
analog-filter=Y
# Digital noise filter bandwidth (0-15)
digital-filter=0
\end{inicode}

\subsection{I2C Commands}

\begin{tabularx}{\textwidth}{p{\fldwcode}Xp{\fldwpld}}
	\toprule
	\textbf{Code} & \textbf{Function} & \textbf{Payload}  \\	
	\midrule	
	
	0 & \flname{WRITE}
	Raw write transaction
	& \makecell[tl]{
		\fldreq
		\fld{u16} slave address \\
		\fld{u8[]} bytes to write \\	
	} \\

	1 & \flname{READ}
	Raw read transaction
	& \makecell[tl]{
		\fldreq
		\fld{u16} slave address \\
		\fld{u16} number of read bytes \\
		\fldresp
		\fld{u8[]} received bytes \\	
	} \\
	
	2 & \flname{WRITE\_REG}
	Write to a slave register. Sends the register number and the data in the same I2C transaction. Multiple registers can be written to slaves supporting auto-increment.
	& \makecell[tl]{
		\fldreq
		\fld{u16} slave address \\
		\fld{u8} register number \\
		\fld{u8[]} bytes to write \\	
	} \\
	
	3 & \flname{READ\_REG}
	Read from a slave register. Writes the register number and issues a read transaction of the given length. Multiple registers can be read from slaves supporting auto-increment.
	& \makecell[tl]{
		\fldreq
		\fld{u16} slave address \\
		\fld{u8} register number \\
		\fld{u16} number of read bytes \\
		\fldresp
		\fld{u8[]} received bytes \\	
	} \\

	\bottomrule
\end{tabularx}









