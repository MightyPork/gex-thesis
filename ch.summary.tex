\chapter{Conclusion}

We have developed an open-source software and hardware platform providing high-level user applications running on a \gls{PC} with access to \gls{GPIO} pins, common hardware buses, and signal acquisition and generation functions.

The platform consists of a FreeRTOS-based firmware for the STM32F072 microcontroller, custom hardware module designs (including realized prototypes), and support software libraries for programming languages C and Python, the latter also compatible with MATLAB. The firmware may be used with the custom hardware, or with existing STM32 development boards.

The devices are connected to the \gls{PC} by one of three interfaces: \gls{USB} as a virtual COM port or with raw endpoint access, a hardware \gls{UART}, or a radio link with the nRF24L01+ transceiver. Configuration is performed by editing INI files exposed in an emulated FAT16 file system through the \gls{USB} connection, or programmatically.

The developed platform can be used as an learning aid, as an inexpensive development tool replacing professional laboratory equipment, or for automation purposes, taking advantage of its hardware interfacing capabilities to connect to multiple sensors and actuators.

Future development should focus on expanding support to other \gls{MCU} models, adding new features, improving the software libraries, and providing a more user-friendly control interface.
